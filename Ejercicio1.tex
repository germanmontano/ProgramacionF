\documentclass{article}

% set font encoding for PDFLaTeX or XeLaTeX
\usepackage{ifxetex}
\ifxetex
  \usepackage{fontspec}
\else
  \usepackage[T1]{fontenc}
  \usepackage[utf8]{inputenc}
  \usepackage{lmodern}
\fi

% used in maketitle
\title{Actividad 1}
\author{Montaño Peraza German Antonio \\
Departamento de Física \\
Universidad de Sonora}
\date{31 de agosto del 2017}

% Enable SageTeX to run SageMath code right inside this LaTeX file.
% documentation: http://mirrors.ctan.org/macros/latex/contrib/sagetex/sagetexpackage.pdf
% \usepackage{sagetex}

\begin{document}
\maketitle
% Son comentarios
\section{Introducción}
En este trabajo investigamos en los links que nos proporcionaba el curso para haberiguar cuales eran los comandos que se podian utilizar en Bash. Para posteriormente practicar y utilizarlos en futuros trabajos.

\subsection{Cosas a destacar}
Durante clase hemos analizado distintas paginas para poder utilizar el sistema latex tanto para trabajos escritos, matematicos, en general trabajos de investigación. De esta manera y bajo practica constante al rededor del semestre lograremos dominar la utilización de esta clase de programas.

\subsubsection{Comandos de Bash}
% Pegamos lo copiado que escribimos en Emacs
\begin{verbatim}
*Dentro de la terminal*
 *ls -l (Permite ver los archivos)
 *ls -al (Permite ver los archivos mas detalladamente)
 *cp (permite copiar los archivos)
 *mkdir (Permite crear carpetas)
 *cd (Cambia de carpeta)
 *emacs (permite acceder a un archivo de texto dentro del porgrama
         emacs)
 *clear (permite limpiar todos los comandos puestos en la terminal)
 *man (abre un manual de la terminal)
 *echo (repite la ultima linea de comando)
 *cd .. (vuelves una carpeta atras)
 *cat (muestra la carpeta en la que te encuentras)
 *history (muestras las acciones realizadas en la terminal)
 *cp -R (copia una carputa con los archivos de su interior)
 *rm (borra los archivos)
 *rm ejemplo* (borra todos los archivos que esten relacionados con lo
               escrito)
 *exit (sales de la terminal, quien lo diria)
 *pwd (indica el directoria actual en el que te encuentras)
 *fire (te dicen que tipo de documentos son)
 *ls -alh (muestra el peso de las carpetas y si este supera los 3
           digitos lo redondea)
 *man -k (muestra todos los manuelas que empiezen por la letra que
         escribas)
 *rmdir (elimina la carpeta que elijas, junto con su interior)
 *touch (cambia fecha y hora del archivo)
 *mv-ejemplo-carpeta en la que estas-../otra carpeta (mueve los
                                                      archivos)0
 *rm (elimina archivos)
 *vi (inserta un comando)
 *cat (permite ver los archivos)
 *menos (permite ver archivos mas grandes)
 *R (Puedes ver el contenido del archivo)
 *W (Puedes cambiar el contenido del archivo)
 *X (Puede ejecutar el archivo o programa)
 *Chmod[permisos][ruta] (permite dar permisos)
 *Ls -Id (permite ver los permisos para un directorio)
 *Head[numero][path] (imprime x numero de lienas)
 *tail[numero][ruta] (imprime x numero de las ultimas lineas)
 *sort[-opcion][ruta] (ordena)
 *Nl[-opciones][ruta] (representa las lineas numericas)
 *cut[-opciones][ruta] (si el contenido esta separado, solo selecciona
                        ciertos campos)
 *sed<expresión>[ruta] (permite buscar y remplazar datos)
 *Uniq[opciones][ruta] (elimina rutas duplicadas)
 *Tac (imprime los datos en orden inverso)
 egrep [linea de comandos]<patrón>[ruta] (busca un determinado
                                         conjunto de datos e imprime
					 cada linea que contenga)
 *> (guarda la salida de un archivo)
 *>> (añade salida a un archivo)
 *< (lee la entrada de un archivo)
 *2> (redirige mensaje de error)
 *l (envia la salida de un programa como entrada de otro)
 *top (muestra todo lo que esta pasando en el sistema)
 *ps (mostrara los procesos de la terminal)
 *ps aux (muestra vista completa del sistema)
 *kill[señal]<PID> (acaba automaticamente  con el programa que le
                   señalemos)
 *jobs (muestra todos los trabajos en un segundo plano)
 *fg (mueve trabajos de segundo plano a un primer plano)
 *Ctrl+z (detiene el programa de primer plano y lo pasa a un segundo
          plano)
 *#! (El asunto.Indica que interprete debe ejecutarse un script)
 *which (le indica el camino a un programa en particular)
 *$ (se coloca antes de un nombre de variable cuando nos referimos a
    su valor)
 *" (se usa para guardar la salida de un programa en una variante)
 *date (muestra la fecha)
 *if [] then else if (realiza una condición)
 *du -sh ./* (mira el tamaño del directorio)
 *df -h (muestra el espacio que ocupa y lo que queda libre)
 
              *Al estar en un edirtor de comandos*
 *ZZ (guardar y salir)
 *Q (quita todos los cambios desde el ultimo guardado)
 *: W (guarda pero no sale)
 *: Wq (de nuevo, guardar y salir)
 
                        *Dentro de emacs*
 *ctrl-x, 1 (se queda con la pantalla donde esté el cursor)
 *ctrl-G (aborta comando)
 *ctrl-x, ctrl-s (salva el archivo)
 *ctrl-x, ctrl-c (sales del archivo)
 
                      *Nacegando en vi*
 *Flechas (mueve el cursor)
 *j,k,h,l (mueve el cursor similar a las flechas)
 *$ (mueve el cursor al final de la linea)
 *NG (se mueve a la n- linea)
 *G (pasar a la ultima linea)
 *W (se mueve al principio de la sig palabra)
 *Nw (se mueve a la n palabra)
 *B (se mueve al principio de la palabra anterior)
 *Nb (mueve de nuevo n palabra)
 *{ (retrocede un párrafo)
 *} (avanza un párrafo)
 *X (elimina un solo caracter)
 *Nx (elimina n caracteres)
 *Dd (borra la linea actual)
 *Dn (borra hasta donde el comando de mov lo lleve)
 *U (desace la última acción)
 *U(Nota:capital) (desace todos los cambios de la linea actual)

               *Conjunto básico de comodines*
 ** (representa 0 o mas caracteres)
 *? (representa un solo carácter)
 *[] (representa un rango de caracteres)
\end{verbatim}

\section{Descripción del trabajo}
Aquí el texto de la segunda sección...
\begin{equation}
F=ma=m\frac{dv}{dt}
\end{equation}
\begin{equation}
W=\int_{a}^{b}F ds
\end{equation}
\section{Conclusion}
Una vez analizado todos los comandos y conseguido distintas paginas para verificar la utilización de otros cuantos, se consiguio dar un paso en la utilización de Bash y el manejo de Latex para entregar trabajos de forma practica.
\end{document}
